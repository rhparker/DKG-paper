\documentclass[12pt]{article}
\usepackage[pdfborder={0 0 0.5 [3 2]}, plainpages=false]{hyperref}%
\usepackage[left=1in,right=1in,top=1in,bottom=1in]{geometry}%
\usepackage[shortalphabetic]{amsrefs}%
\usepackage{amsmath}
\usepackage{enumerate}
% \usepackage{enumitem}
\usepackage{amssymb}                
\usepackage{amsmath}                
\usepackage{amsfonts}
\usepackage{amsthm}
\usepackage{bbm}
\usepackage[table,xcdraw]{xcolor}
\usepackage{tikz}
\usepackage{float}
\usepackage{booktabs}
\usepackage{svg}
\usepackage{mathtools}
\usepackage{cool}
\usepackage{url}
\usepackage{graphicx,epsfig}
\usepackage{makecell}
\usepackage{array}

\usepackage[capitalize,nameinlink]{cleveref}
% Per SIAM Style Manual, "section" should be lowercase
\crefname{section}{section}{sections}
\crefname{subsection}{subsection}{subsections}
\Crefname{section}{Section}{Sections}
\Crefname{subsection}{Subsection}{Subsections}

% Per SIAM Style Manual, "Figure" should be spelled out in references
\Crefname{figure}{Figure}{Figures}

% Per SIAM Style Manual, don't say equation in front on an equation.
\crefformat{equation}{\textup{#2(#1)#3}}
\crefrangeformat{equation}{\textup{#3(#1)#4--#5(#2)#6}}
\crefmultiformat{equation}{\textup{#2(#1)#3}}{ and \textup{#2(#1)#3}}
{, \textup{#2(#1)#3}}{, and \textup{#2(#1)#3}}
\crefrangemultiformat{equation}{\textup{#3(#1)#4--#5(#2)#6}}%
{ and \textup{#3(#1)#4--#5(#2)#6}}{, \textup{#3(#1)#4--#5(#2)#6}}{, and \textup{#3(#1)#4--#5(#2)#6}}

% But spell it out at the beginning of a sentence.
\Crefformat{equation}{#2Equation~\textup{(#1)}#3}
\Crefrangeformat{equation}{Equations~\textup{#3(#1)#4--#5(#2)#6}}
\Crefmultiformat{equation}{Equations~\textup{#2(#1)#3}}{ and \textup{#2(#1)#3}}
{, \textup{#2(#1)#3}}{, and \textup{#2(#1)#3}}
\Crefrangemultiformat{equation}{Equations~\textup{#3(#1)#4--#5(#2)#6}}%
{ and \textup{#3(#1)#4--#5(#2)#6}}{, \textup{#3(#1)#4--#5(#2)#6}}{, and \textup{#3(#1)#4--#5(#2)#6}}

% Make number non-italic in any environment.
\crefdefaultlabelformat{#2\textup{#1}#3}

\def\noi{\noindent}
\def\T{{\mathbb T}}
\def\R{{\mathbb R}}
\def\N{{\mathbb N}}
\def\C{{\mathbb C}}
\def\Z{{\mathbb Z}}
\def\P{{\mathbb P}}
\def\E{{\mathbb E}}
\def\Q{\mathbb{Q}}
\def\ind{{\mathbb I}}

\DeclareMathOperator{\spn}{span}
\DeclareMathOperator{\ran}{range}

\graphicspath{ {images/} }

\newtheorem{lemma}{Lemma}
\newtheorem{theorem}{Theorem}
\newtheorem{corollary}{Corollary}
\newtheorem{definition}{Definition}
\newtheorem{proposition}{Proposition}
\newtheorem{hypothesis}{Hypothesis}

\newtheorem{notation}{Notation}

\begin{document}

\section{Kink-antikinks in discrete Klein Gordon}

Main references are \cite{KevrekidisWeinstein2000} and \cite{Kapitula2001}. The discrete Klein-Gordon equation (DKG) is given by
\begin{equation}\label{eq:KG}
\ddot{u_n} = d (\Delta_2 u)_n - f(u_n)
\end{equation}
where $\Delta_2$ is the discrete second difference operator and $f(u) = V'(u)$ for a smooth potential function $V(u)$. Versions we are interested in are discrete sine-Gordon, where $f(u) = -\sin(u)$, and $\phi^4$ variant, where $f(u) = -u(1-u^2)$. Importantly, $f(u)$ satisfies the following properties.
\begin{itemize}
	\item $f(u)$ is an odd function (so $f(0) = 0$) and $f'(0) < 0$
	\item there is a pair of nonzero equilibria $\pm u^*$ with $f(\pm u^*) = 0$ and $f'(u^*) = f'(-u^*) > 0$
	\item there are no other equilibria in $[-u^*, u^*]$
\end{itemize}

Equilibrium solutions (standing waves) satisfy 
\begin{equation}\label{eq:KGeq}
d (\Delta_2 u)_n - f(u_n) = 0
\end{equation}

Linearization about an solution $u_n$ (kink, etc.) give us the eigenvalue problem
\[
d (\Delta_2 v)_n - f'(u_n)v_n = \lambda^2 v_n
\]
Considered as an eigenvalue problem for $\lambda^2$, this is self-adjoint (symmetric, infinite dimensional matrix), so $\lambda^2$ is always real, which means that $\lambda$ must be either real or pure imaginary. Taking $\omega = \lambda^2$, we write the eigenvalue problem as
\begin{equation}\label{eq:evp}
d (\Delta_2 v)_n - f'(u_n)v_n = \omega v_n,
\end{equation}
where the eigenvalues are $\lambda = \pm \sqrt{\omega}$.

Using a spatial dynamics approach, let $U(n) = (u(n), \tilde{u}(n)) = (u_n, u_{n-1})$. Then this is equivalent to the lattice dynamical system in $\R^2$
\begin{equation}\label{eq:dynEq}
U(n+1) = F(U(n)),
\end{equation}
where
\[
F\begin{pmatrix}u \\ \tilde{u} \end{pmatrix} =
\begin{pmatrix}2u - \tilde{u} + \frac{1}{d}f(u) \\
u
\end{pmatrix}
\]
This map has three fixed points $(0,0)$ and $(\pm u^*, \pm u^*)$. For convenience, let $S^\pm = (\pm u^*, \pm u^*)^T$. Linearizing about the fixed points $S^\pm$, we obtain the matrix 
\[
D F(S^\pm)=
\begin{pmatrix}2 + \frac{1}{d}f'(\pm u^*) & -1 \\ 1 & 0
\end{pmatrix}
\]
Since $f'(\pm u^*) > 0$, this has a pair of eigenvalues $\{ r, 1/r \}$ with $r > 0$, where
\[
r = \frac{1}{2d}\left( f'(u^*) + 2d + \sqrt{f'(u^*)(f'(u^*) + 4d)} \right)
\]

Thus $S^\pm$ are hyperbolic saddle equilibria of the lattice dynamical system \cref{eq:dynEq}. We take the existence of a symmetric kink as a hypothesis. From the spatial dynamics perspective, this is a heteroclinic orbit connecting the saddle at $(-u^*, -u^*)$ to the saddle at $(u^*, -u^*)$.

\begin{hypothesis}\label{hyp:kinkexists}
There exists a kink solution $K(n)$ to \cref{eq:dynEq} which connects the unstable manifold $W^u(-u^*, -u^*)$ and the stable manifold $W^s(u^*, u^*)$. These manifolds intersect transversely in $\R^2$. The kink solution has one of the following odd symmetries:
\begin{itemize}
	\item $K(-n) = -K(n-1)$ (``intersite'' symmetry)
	\item $K(-n) = -K(n)$ (``onsite'' symmetry)
\end{itemize}
\end{hypothesis}

Since $f(u)$ is an odd function, if $u_n$ is a solution to \cref{eq:KGeq}, so is $-u_n$. Thus for every kink solution $K(n)$ to \cref{eq:dynEq} there is a corresponding antikink solution $\tilde{K}(n) = -K(n)$.

We can now construct multi-kink solutions by joining together kinks and anti-kinks end-to-end. Given the symmetry just discussed, we can always without loss of generality start with a kink solution.

We will characterize a multi-kink in the following way. Let $m > 1$ be the total number of kinks and antikinks. Let $N_i$ ($i = 1, \dots, m-1$) be the distances (in lattice points) between consecutive kinks/anti-kinks. We seek a solution which can be written piecewise in the form 
\begin{equation}\label{Upiecewise}
\begin{aligned}
U_i^-(n) &= c_i K(n) + \tilde{K}_i^-(n) && n \in [-N_{i-1}^-, 0] \\
U_i^+(n) &= c_i K(n) + \tilde{K}_i^+(n) && n \in [0, N_i^+],
\end{aligned}
\end{equation}
where $c_i = (-1)^{i+1}$, $N_i^+ = \lfloor \frac{N_i}{2} \rfloor$, $N_i^- = N_i - N_i^+$, $N_0^- = N_m^+ = \infty$, and
\begin{equation}\label{defN}
N = \frac{1}{2} \min\{ N_i \}.
\end{equation}
The individual pieces are joined together end-to-end as in \cite{Sandstede1998}. The functions $\tilde{K}_i^\pm(n)$ are remainder terms, which we expect to be small. We then have the following theorem.

\begin{theorem}\label{th:KaKexists}
Assume \cref{hyp:kinkexists}. Then there exists a positive integer $N_0$ with the following property. For all $m > 1$ and distances $N_i \geq N_0$, there exists a unique solution $U(n)$ which is composed of $m$ alternating kinks and anti-kinks and can be written piecewise in the form \cref{Upiecewise}. For the remainder terms $\tilde{K}_i^\pm(n)$, we have the estimates
\begin{equation}\label{eq:Westimates}
\begin{aligned}
\|\tilde{K}_i^\pm\| &\leq C r^{-N} \\
\tilde{K}_i^+(N_i^+) &= c_{i+1} K(-N_i^-) + \mathcal{O}(r^{-2N}) \\
\tilde{K}_{i+1}^-(-N_i^-) &= c_i K(N_i^+) + \mathcal{O}(r^{-2N}) .
\end{aligned}
\end{equation}
WE WILL ACTUALLY NEED BETTER ESTIMATES BUT I KNOW HOW TO DO IT.
\end{theorem}

For spectral stability, rewrite the eigenvalue problem \cref{eq:evp} as a lattice dynamical system by taking $V(n) = (v(n), \tilde{v}(n)) = (v_n, v_{n-1})$. Then \cref{eq:evp} is equivalent to the lattice dynamical system in $\R^2$
\begin{equation}\label{eq:EVPdyneq}
V(n+1) = D F( U(n) )V(n) + \omega B V(n),
\end{equation}
where
\[
B = \frac{1}{d}
\begin{pmatrix}1 & 0 \\ 0 & 0
\end{pmatrix}
\] 

\section{Proof of existence results}

THIS WILL NEED TO BE REDONE A BIT SINCE WE WILL NEED BETTER ESTIMATES THAN ARE IN \cite{Parker2020}. SHOULD BE STRAIGHTFORWARD.

First, we rewrite the system as a fixed point problem. Expanding $F(u)$ in a Taylor series about $c_i K(n)$, we get
\begin{align*}
F(U_i^\pm(n)) &= F(c_i K(n) + \tilde{K}_i^-(n)) = 
D F(c_i K(n)) \tilde{K}_i^\pm(n) + G(\tilde{K}_i^\pm(n)),
\end{align*}
where $G(\tilde{K}_i^\pm(n)) = \mathcal{O}(|\tilde{K}_i^\pm|^2)$ with $G(0) = 0$ and $DG(0) = 0$. Since $f(u)$ is even, $f'(u)$ is odd, thus $D F(c_i K(n)) = D F(K(n))$, and so this becomes
\begin{align*}
F(U_i^\pm(n)) &= 
D F(K(n)) \tilde{K}_i^\pm(n) + G(\tilde{K}_i^\pm(n)),
\end{align*}
Let
\begin{align}\label{didef}
d_i &= c_{i+1} K(-N_i^-) - c_i K(N_i^+).
\end{align}
Substituting these, we obtain the following system of equations for the remainder functions $\tilde{K}_i^\pm$.
\begin{align}
\tilde{K}_i^\pm(n+1) &= D F(K(n)) \tilde{K}_i^\pm(n) + G(\tilde{K}_i^\pm(n)) \label{eq:Wsystem1} \\
\tilde{K}_i^+(N_i^+) - \tilde{K}_{i+1}^-(-N_i^-) &= d_i \label{eq:Wsystem2} \\
\tilde{K}_i^+(0) - \tilde{K}_i^-(0) &= 0. \label{eq:Wsystem3}
\end{align}
Let $\Phi(m, n)$ be the evolution operators for the linear difference equation 
\[
V(n+1) = D F(K(n)) V(n).
\]
By the stable manifold theorem, $|K(n) - S^+| \leq C r^{-|n|}$ for $n \geq 0$ and $|K(n) - S^-| \leq C r^{-|n|}$ for $n \leq 0$. Since $S^\pm$ are hyperbolic, we can decompose the evolution operator $\Phi(m, n)$ in exponential dichotomies on $\Z^\pm$. The proof then follows that of \cite[Theorem 3]{Parker2020}. Briefly, we write equation \cref{eq:Wsystem1} in fixed-point form using the discrete variation of constants formula together with projections on the stable and unstable subspaces of the exponential dichotomy. As long as $N$ is sufficiently large, we use the implicit function theorem to solve for the remainder functions $\tilde{K}_i^\pm$ as well as the matching conditions \cref{eq:Wsystem2} and \cref{eq:Wsystem3}. Since $W^u(S^-)$ and $W^s(S^+)$ intersect transversely, we have the decomposition $\R^2 = T_{K(0)}W^u(S^-)\oplus T_{K(0)}W^s(S^+)$. As in the proof of \cite[Theorem 3]{Parker2020}, solving \cref{eq:Wsystem3} does not involve jump conditions.

\section{Proof of stability results}

For the primary kink $K(n) = (k_n, \tilde{k}_n)$, let $v_0(n)$ be an eigenfunction with corresponding eigenvalue $\lambda_0$, let $\omega_0 = \lambda_0^2$, and let $V_0(n) = (v_0(n), \tilde{v}_0(n)) = (v_0(n), v_0(n-1))$ so that $V_0(n)$ solves the equation
\begin{equation}\label{eq:V0solves}
V_0(n+1) = D F( K(n) )V_0(n) + \omega_0 B V_0(n)
\end{equation}
For the kink-antikink solution $U(n)$, we will take an ansatz which is a piecewise perturbation of $V_0(n)$. Let
\begin{equation}\label{eq:Viansatz}
V_i^\pm(n) = d_i c_i V_0(n) + W_i^\pm(n),
\end{equation}
where $d_i \in \C$ and $c_i = (-1)^{i+1}$. Substituting this into \cref{eq:EVPdyneq} and simplifying using \cref{eq:V0solves}, the eigenvalue problem becomes
\begin{equation}\label{eq:Weq1}
\begin{aligned}
W_i^\pm(n+1)
&= DF(c_i K(n)) W_i^\pm(n) + \omega_0 B W_i^\pm(n) \\
&\qquad + [G_i^\pm(n) + (\omega - \omega_0) B](d_i c_i V_0(n) W_i^\pm(n))
\end{aligned}
\end{equation}
where
\[
G_i^\pm(n) = DF(U_i^\pm(n)) - DF(c_i K(n)).
\]
Let 
\[
A_i(n; \omega_0) = DF(c_i K(n)) + \omega_0 B
\]
Then \cref{eq:Weq1} becomes
\begin{align}\label{eq:Weq2}
W_i^\pm(n+1)
&= A_i(n; \omega_0) W_i^\pm(n) + [G_i^\pm(n) + (\omega - \omega_0) B](d_i c_i V_0(n) + W_i^\pm(n)).
\end{align}

As long as $|\omega_0| < f'(\pm u*)$ (WHICH WE SHOULD BE ABLE TO SHOW OR CAN ASSUME, IN ANY CASE IT IS REASONABLE GIVEN TYPICAL $f)$), $A_i(n)$ is exponentially asymptotic to a constant matrix with an exponential dichotomy, so it has one as well. WE CAN NOW USE ALL THE EXPONENTIAL DICHOTOMY TRICKS WE KNOW.

In addition to solving \cref{eq:Weq2}, the eigenfunction must satisfy matching conditions at $n = \pm N_i$ and $n = 0$. Thus the system of equations we need to solve is
\begin{equation}\label{eq:eigWsystem1}
\begin{aligned}
& W_i^\pm(n+1)
= A_i(n; \omega_0) W_i^\pm(n) + [G_i^\pm(n) + (\omega - \omega_0) B](d_i c_i V_0(n) + W_i^\pm(n))\\
& W_i^+(N_i^+) - W_{i+1}^-(-N_i^-) &= D_i d \\
& W_i^+(0) - W_i^-(0) = 0,
\end{aligned}
\end{equation}
where
\begin{equation}\label{defDid}
D_i d = d_{i+1} c_{i+1} V_0(-N_i^-) - d_i c_i V_0(N_i^+)
\end{equation}

Things to note:
\begin{itemize}
\item Equation $V(n+1) = A(n; \lambda_0)$ has unique bounded solution $V_0(n) = (v_0(n), v_0(n-1))$. Let $\Phi(m, n; \lambda_0)$ be the evolution operator for this.
\item Adjoint equation $Z(n) = A(n; \lambda_0)^* Z(n+1)$ has unique bounded solution $Z_0(n) = (-v_0(n-1), v_0(n))$ which is orthogonal to $Z_0(n)$. (This uses the fact that $\lambda_0$ is real or pure imaginary when we take the adjoint).
\item $A(n; \lambda_0)$ is exponentially asymptotic (on both ends) to a hyperbolic matrix $A_0$ with 1-dimensional stable and unstable eigenspaces $E^s$ and $E^u$ with corresponding projections $P_0^s$ and $P_0^u$.
\item Because of this, $V(n+1) = A(n; \lambda_0)$ induces an exponential dichotomy on $\Z^\pm$.
\end{itemize}

Since $\R^2 = \spn\{ V_0(0), Z_0(0) \}$, we can rewrite our system as 

\begin{equation}\label{eq:eigWsystem2}
\begin{aligned}
&W_i^\pm(n+1)
= A_i(n; \omega_0) W_i^\pm(n) + [G_i^\pm(n) + (\omega - \omega_0) B](d_i c_i V_0(n) + W_i^\pm(n))\\
&W_i^+(N_i^+) - W_{i+1}^-(-N_i^-) = D_i d \\
&W_i^+(0) - W_i^-(0) \in \C Z_0(n),
\end{aligned}
\end{equation}
with jump conditions
\begin{equation}
\langle Z_0(n), W_i^+(0) - W_i^-(0) \rangle = 0
\end{equation}

As in \cite{Sandstede1998}, we write equation \cref{eq:Weq1} as a fixed point problem using the discrete variation of constants formula together with projections on the stable and unstable subspaces of the exponential dichotomy. Let $\delta > 0$ be small, and choose $N$ sufficiently large so that $r^{-N} < \delta$. Define the spaces
\begin{align*}
V_W &= \ell^\infty([-N_{i-1}, 0]) \oplus \ell^\infty([0, N_i])  \\
V_a &= \bigoplus_{i=0}^{n-1} E^u \oplus E^s \\
V_b &= \bigoplus_{i=0}^{n-1} \ran P_-^u(0) \oplus \ran P_+^s(0)\\
V_\lambda &= B_\delta(\lambda_0) \subset \C \\
V_d &= \C^d.
\end{align*}
Then for
\begin{align*}
W = (W_i^-, W_i^+) &\in V_W \\
a = (a_i^-, a_i^+) &\in V_a \\
b = (b_i^-, b_i^+) &\in V_b \\
\lambda &\in V_\lambda,
\end{align*}
the fixed point equations for the eigenvalue problem are
\begin{equation}\label{fpeig}
\begin{aligned}
W_i^-(n) &= 
\Phi_s^-(n, -N_{i-1}^-) a_{i-1}^- + \sum_{j = -N_{i-1}^-}^{n-1} \Phi_s^-(n, j+1)
[G_i^-(j) + (\omega - \omega_0) B](d_i c_i V_0(j) + W_i^-(j))
 \\
&+ \Phi_u^-(n, 0) b_i^- - \sum_{j = n}^{-1} \Phi_u^-(n, j+1) 
[G_i^-(j) + (\omega - \omega_0) B](d_i c_i V_0(j) + W_i^-(j))\\
W_i^+(n) &= \Phi_s^+(n, 0; \theta_i) b_i^+ + \sum_{j = 0}^{n-1} \Phi_s^+(n, j+1) 
[G_i^+(j) + (\omega - \omega_0) B](d_i c_i V_0(j) + W_i^+(j))\\
&+ \Phi_u^+(n, N_i^+) a_i^+ - \sum_{j = n}^{N_i^+-1} \Phi_u^+(n, j+1) 
[G_i^+(j) + (\omega - \omega_0) B](d_i c_i V_0(j) + W_i^+(j)),
\end{aligned}
\end{equation}
where $a_0^- = a_m^+ = 0$ and the sums are defined to be $0$ if the upper index is smaller than the lower index.

We now follow the procedure and invert the system \cref{eq:eigWsystem2} in the following steps.

\begin{enumerate}
\item Solve for the remainder functions $W_i^\pm(n)$
\item Use the second equation in \cref{eq:eigWsystem2} to solve for the $a_i^\pm$
\item Use the third equation in \cref{eq:eigWsystem2} to solve for the $b_i^\pm$
\item Evaluate the jump conditions
\end{enumerate}

At the end of the day, we should have
\begin{align*}
a_i^+ &= P_0^u D_i d + \text{``h.o.t.''} \\
a_i^- &= -P_0^s D_i d + \text{``h.o.t.''}
\end{align*}
When we compute the jump conditions, the only terms that should contribute are the ones with $a_i^\pm$ (tail interactions) and the sum involving $(\omega - \omega_0) B (d_i c_i V_0(j)$ (Melnikov-type sum). Putting all of this together, the jump conditions should be, to leading order, 
\begin{equation}\label{eq:xieq}
\begin{aligned}
\xi_i = \langle &Z_0(N_i^+), P_0^u D_i d \rangle 
+ \langle Z_0(-N_{i-1}^-), P_0^s D_{i-1} d \rangle 
- (\omega - \omega_0) d_i c_i \sum_{j = -\infty}^{\infty} \langle Z_0(j+1), B V_0(j)\rangle
\end{aligned}
\end{equation}
For the terms in the Melnikov sum,
\[
\langle Z_0(j+1), B V_0(j)\rangle = \frac{1}{d} \langle (-v_0(j), v_0(j-1))^T, (v_0(j), 0) \rangle
= -\frac{1}{d}v_0(j)^2,
\]
which means the Melnikov sum will be given by
\begin{equation}\label{eq:M}
M = \sum_{j = -\infty}^{\infty} v_0(j)^2 = \| v_0 \|_{\ell^2},
\end{equation}
which is always positive. For the other terms, we note that
\begin{align*}
P_0^u D_i d &= d_{i+1} c_{i+1} V_0(-N_i^-) + \text{``h.o.t.''} \\
P_0^s D_i d &= -d_i c_i V_0(N_i^+) + \text{``h.o.t.''}
\end{align*}
Putting all of this together, we have, to leading order,
\begin{align*}
\xi_i = \langle Z_0(N_i^+), V_0(-N_i^-) \rangle d_{i+1} c_{i+1}
- \langle Z_0(-N_{i-1}^-), V_0(N_{i-1}^+) \rangle d_{i-1} c_{i-1}
+ \dfrac{1}{d} (\omega - \omega_0) d_i c_i M
\end{align*}
Since $c_i$ and $c_{i+1}$ have opposite signs, this simplifies to 
\begin{align*}
\xi_i = \langle Z_0(N_i^+), V_0(-N_i^-) \rangle d_{i+1}
- \langle Z_0(-N_{i-1}^-), V_0(N_{i-1}^+) \rangle d_{i-1}
- \dfrac{1}{d} (\omega - \omega_0) d_i M
\end{align*}
Let
\begin{align*}
a_i &= \langle Z_0(N_i^+), V_0(-N_i^-) \rangle 
= v_0(N_i^+)v_0(-N_i^- - 1) - v_0(N_i^+ - 1)v_0(-N_i^-)\\
\tilde{a}_i &= \langle Z_0(-N_i^-), V_0(N_i^+) \rangle 
= v_0(N_i^+ - 1)v_0(-N_i^-) - v_0(N_i^+)v_0(-N_i^- - 1)
\end{align*}
These can be simplified based on symmetries of the eigenfunction $v_0(n)$. For even eigenfunctions, we have $v_0(-n) = v_0(n-1)$, (the Goldstone mode, appears to be even), these become
\begin{align*}
a_i &= \langle Z_0(N_i^+), V_0(-N_i^-) \rangle 
= v_0(N_i^+)v_0(-N_i^- - 1) - v_0(N_i^+ - 1)v_0(-N_i^-)\\
\tilde{a}_i &= \langle Z_0(-N_i^-), V_0(N_i^+) \rangle 
= v_0(N_i^+ - 1)v_0(-N_i^-) - v_0(N_i^+)v_0(-N_i^- - 1)
\end{align*}

So then we have
\begin{align*}
\xi_i = a_i d_{i+1}
- \tilde{a}_{i-1} d_{i-1}
- \dfrac{1}{d} (\lambda^2 - \lambda_0^2) d_i M
\end{align*}





\bibliographystyle{amsplain}
\bibliography{DKG.bib}




\end{document}